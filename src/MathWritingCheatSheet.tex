\documentclass[10pt,letterpaper,oneside,reqno]{amsart}
\usepackage{amsfonts}
\usepackage{amsmath}
\usepackage{amssymb}
\usepackage{amsthm}
\usepackage{mathrsfs}
\usepackage[left=1in,right=1in,bottom=1in,top=1in]{geometry}
\usepackage[pdfpagelabels,hyperindex,colorlinks=true,linkcolor=blue,urlcolor=magenta,citecolor=green]{hyperref}
\usepackage{graphicx}
\emergencystretch=1em
\usepackage{array}
\usepackage{multicol}
\usepackage{etoolbox}
\apptocmd{\sloppy}{\hbadness 10000\relax}{}{}
\raggedbottom

\newtheorem{theorem}{Theorem}[section]
\newtheorem{corollary}[theorem]{Corollary}
\newtheorem{lemma}[theorem]{Lemma}
\newtheorem{example}[theorem]{Example}
\newtheorem{conjecture}[theorem]{Conjecture}
\newtheorem{definition}[theorem]{Definition}

\title[Mathematical writing Cheat Sheet]
{Mathematical writing Cheat Sheet}
\hypersetup{
    pdftitle={Mathematical writing Cheat Sheet},
    pdfsubject={
        Mathematical writing
    },
    pdfauthor={Petro Kolosov},
    pdfkeywords={
        Mathematical writing
    }
}
\begin{document}

    \maketitle

    \begin{multicols}{2}

        \section*{Logical Progression and Reasoning}
        \begin{itemize}
            \item \textit{Therefore, it follows that...}
            \item \textit{Thus, we conclude that...}
            \item \textit{Hence, we obtain...}
            \item \textit{From this, it follows that...}
            \item \textit{As a result, we deduce that...}
            \item \textit{Consequently, we see that...}
        \end{itemize}

        \section*{If-Then Statements and Assumptions}
        \begin{itemize}
            \item \textit{If \( X \) holds, then \( Y \) must also hold.}
            \item \textit{Suppose that \( P \) is true; then we must have...}
            \item \textit{Given that \( A \) is satisfied, it follows that...}
            \item \textit{Assume that \( f(x) \) is differentiable; then we can write...}
            \item \textit{Under these conditions, we can conclude that...}
        \end{itemize}

        \section*{Definitions and Explanations}
        \begin{itemize}
            \item \textit{We define \( f(x) \) as follows:}
            \item \textit{The term "X" refers to...}
            \item \textit{By definition, we have...}
            \item \textit{Formally, a function is said to be continuous if...}
            \item \textit{For the sake of clarity, we introduce the notation...}
        \end{itemize}

        \section*{Proofs and Justifications}
        \begin{itemize}
            \item \textit{To prove this, we proceed as follows...}
            \item \textit{We now establish the claim by induction.}
            \item \textit{Consider the case where...}
            \item \textit{By contradiction, suppose that...}
            \item \textit{This result follows directly from Theorem X.}
            \item \textit{Applying Lemma Y, we obtain...}
            \item \textit{Using the assumption that..., we see that...}
        \end{itemize}

        \columnbreak

        \section*{Comparisons and Contrasts}
        \begin{itemize}
            \item \textit{Unlike the previous case, here we find that...}
            \item \textit{This result is similar to... but differs in that...}
            \item \textit{In contrast to..., we now observe that...}
            \item \textit{A key distinction between these cases is that...}
            \item \textit{While \( X \) holds in general, it does not necessarily imply \( Y \).}
        \end{itemize}

        \section*{Example and Counterexample}
        \begin{itemize}
            \item \textit{As an example, consider the function...}
            \item \textit{For instance, if we take \( x = 2 \), then...}
            \item \textit{A simple case to illustrate this is...}
            \item \textit{However, the following counterexample shows that...}
            \item \textit{To demonstrate that this condition is necessary, consider...}
        \end{itemize}

        \section*{Summarizing and Concluding}
        \begin{itemize}
            \item \textit{In summary, we have shown that...}
            \item \textit{To conclude, we have established that...}
            \item \textit{This completes the proof of Theorem X.}
            \item \textit{The main result can be summarized as follows...}
            \item \textit{Overall, these findings demonstrate that...}
        \end{itemize}

        \section*{Transitions Between Steps}
        \begin{itemize}
            \item \textit{Next, we consider the case where...}
            \item \textit{Proceeding in a similar manner, we obtain...}
            \item \textit{We now turn our attention to...}
            \item \textit{Applying the previous result, we get...}
            \item \textit{Rewriting the equation, we find that...}
        \end{itemize}

    \end{multicols}


\end{document}
